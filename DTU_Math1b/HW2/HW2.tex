\documentclass[11pt]{article}

    \usepackage[breakable]{tcolorbox}
    \usepackage{parskip} % Stop auto-indenting (to mimic markdown behaviour)
    

    % Basic figure setup, for now with no caption control since it's done
    % automatically by Pandoc (which extracts ![](path) syntax from Markdown).
    \usepackage{graphicx}
    % Keep aspect ratio if custom image width or height is specified
    \setkeys{Gin}{keepaspectratio}
    % Maintain compatibility with old templates. Remove in nbconvert 6.0
    \let\Oldincludegraphics\includegraphics
    % Ensure that by default, figures have no caption (until we provide a
    % proper Figure object with a Caption API and a way to capture that
    % in the conversion process - todo).
    \usepackage{caption}
    \DeclareCaptionFormat{nocaption}{}
    \captionsetup{format=nocaption,aboveskip=0pt,belowskip=0pt}

    \usepackage{float}
    \floatplacement{figure}{H} % forces figures to be placed at the correct location
    \usepackage{xcolor} % Allow colors to be defined
    \usepackage{enumerate} % Needed for markdown enumerations to work
    \usepackage{geometry} % Used to adjust the document margins
    \usepackage{amsmath} % Equations
    \usepackage{amssymb} % Equations
    \usepackage{textcomp} % defines textquotesingle
    % Hack from http://tex.stackexchange.com/a/47451/13684:
    \AtBeginDocument{%
        \def\PYZsq{\textquotesingle}% Upright quotes in Pygmentized code
    }
    \usepackage{upquote} % Upright quotes for verbatim code
    \usepackage{eurosym} % defines \euro

    \usepackage{iftex}
    \ifPDFTeX
        \usepackage[T1]{fontenc}
        \IfFileExists{alphabeta.sty}{
              \usepackage{alphabeta}
          }{
              \usepackage[mathletters]{ucs}
              \usepackage[utf8x]{inputenc}
          }
    \else
        \usepackage{fontspec}
        \usepackage{unicode-math}
    \fi

    \usepackage{fancyvrb} % verbatim replacement that allows latex
    \usepackage{grffile} % extends the file name processing of package graphics
                         % to support a larger range
    \makeatletter % fix for old versions of grffile with XeLaTeX
    \@ifpackagelater{grffile}{2019/11/01}
    {
      % Do nothing on new versions
    }
    {
      \def\Gread@@xetex#1{%
        \IfFileExists{"\Gin@base".bb}%
        {\Gread@eps{\Gin@base.bb}}%
        {\Gread@@xetex@aux#1}%
      }
    }
    \makeatother
    \usepackage[Export]{adjustbox} % Used to constrain images to a maximum size
    \adjustboxset{max size={0.9\linewidth}{0.9\paperheight}}

    % The hyperref package gives us a pdf with properly built
    % internal navigation ('pdf bookmarks' for the table of contents,
    % internal cross-reference links, web links for URLs, etc.)
    \usepackage{hyperref}
    % The default LaTeX title has an obnoxious amount of whitespace. By default,
    % titling removes some of it. It also provides customization options.
    \usepackage{titling}
    \usepackage{longtable} % longtable support required by pandoc >1.10
    \usepackage{booktabs}  % table support for pandoc > 1.12.2
    \usepackage{array}     % table support for pandoc >= 2.11.3
    \usepackage{calc}      % table minipage width calculation for pandoc >= 2.11.1
    \usepackage[inline]{enumitem} % IRkernel/repr support (it uses the enumerate* environment)
    \usepackage[normalem]{ulem} % ulem is needed to support strikethroughs (\sout)
                                % normalem makes italics be italics, not underlines
    \usepackage{soul}      % strikethrough (\st) support for pandoc >= 3.0.0
    \usepackage{mathrsfs}
    

    
    % Colors for the hyperref package
    \definecolor{urlcolor}{rgb}{0,.145,.698}
    \definecolor{linkcolor}{rgb}{.71,0.21,0.01}
    \definecolor{citecolor}{rgb}{.12,.54,.11}

    % ANSI colors
    \definecolor{ansi-black}{HTML}{3E424D}
    \definecolor{ansi-black-intense}{HTML}{282C36}
    \definecolor{ansi-red}{HTML}{E75C58}
    \definecolor{ansi-red-intense}{HTML}{B22B31}
    \definecolor{ansi-green}{HTML}{00A250}
    \definecolor{ansi-green-intense}{HTML}{007427}
    \definecolor{ansi-yellow}{HTML}{DDB62B}
    \definecolor{ansi-yellow-intense}{HTML}{B27D12}
    \definecolor{ansi-blue}{HTML}{208FFB}
    \definecolor{ansi-blue-intense}{HTML}{0065CA}
    \definecolor{ansi-magenta}{HTML}{D160C4}
    \definecolor{ansi-magenta-intense}{HTML}{A03196}
    \definecolor{ansi-cyan}{HTML}{60C6C8}
    \definecolor{ansi-cyan-intense}{HTML}{258F8F}
    \definecolor{ansi-white}{HTML}{C5C1B4}
    \definecolor{ansi-white-intense}{HTML}{A1A6B2}
    \definecolor{ansi-default-inverse-fg}{HTML}{FFFFFF}
    \definecolor{ansi-default-inverse-bg}{HTML}{000000}

    % common color for the border for error outputs.
    \definecolor{outerrorbackground}{HTML}{FFDFDF}

    % commands and environments needed by pandoc snippets
    % extracted from the output of `pandoc -s`
    \providecommand{\tightlist}{%
      \setlength{\itemsep}{0pt}\setlength{\parskip}{0pt}}
    \DefineVerbatimEnvironment{Highlighting}{Verbatim}{commandchars=\\\{\}}
    % Add ',fontsize=\small' for more characters per line
    \newenvironment{Shaded}{}{}
    \newcommand{\KeywordTok}[1]{\textcolor[rgb]{0.00,0.44,0.13}{\textbf{{#1}}}}
    \newcommand{\DataTypeTok}[1]{\textcolor[rgb]{0.56,0.13,0.00}{{#1}}}
    \newcommand{\DecValTok}[1]{\textcolor[rgb]{0.25,0.63,0.44}{{#1}}}
    \newcommand{\BaseNTok}[1]{\textcolor[rgb]{0.25,0.63,0.44}{{#1}}}
    \newcommand{\FloatTok}[1]{\textcolor[rgb]{0.25,0.63,0.44}{{#1}}}
    \newcommand{\CharTok}[1]{\textcolor[rgb]{0.25,0.44,0.63}{{#1}}}
    \newcommand{\StringTok}[1]{\textcolor[rgb]{0.25,0.44,0.63}{{#1}}}
    \newcommand{\CommentTok}[1]{\textcolor[rgb]{0.38,0.63,0.69}{\textit{{#1}}}}
    \newcommand{\OtherTok}[1]{\textcolor[rgb]{0.00,0.44,0.13}{{#1}}}
    \newcommand{\AlertTok}[1]{\textcolor[rgb]{1.00,0.00,0.00}{\textbf{{#1}}}}
    \newcommand{\FunctionTok}[1]{\textcolor[rgb]{0.02,0.16,0.49}{{#1}}}
    \newcommand{\RegionMarkerTok}[1]{{#1}}
    \newcommand{\ErrorTok}[1]{\textcolor[rgb]{1.00,0.00,0.00}{\textbf{{#1}}}}
    \newcommand{\NormalTok}[1]{{#1}}

    % Additional commands for more recent versions of Pandoc
    \newcommand{\ConstantTok}[1]{\textcolor[rgb]{0.53,0.00,0.00}{{#1}}}
    \newcommand{\SpecialCharTok}[1]{\textcolor[rgb]{0.25,0.44,0.63}{{#1}}}
    \newcommand{\VerbatimStringTok}[1]{\textcolor[rgb]{0.25,0.44,0.63}{{#1}}}
    \newcommand{\SpecialStringTok}[1]{\textcolor[rgb]{0.73,0.40,0.53}{{#1}}}
    \newcommand{\ImportTok}[1]{{#1}}
    \newcommand{\DocumentationTok}[1]{\textcolor[rgb]{0.73,0.13,0.13}{\textit{{#1}}}}
    \newcommand{\AnnotationTok}[1]{\textcolor[rgb]{0.38,0.63,0.69}{\textbf{\textit{{#1}}}}}
    \newcommand{\CommentVarTok}[1]{\textcolor[rgb]{0.38,0.63,0.69}{\textbf{\textit{{#1}}}}}
    \newcommand{\VariableTok}[1]{\textcolor[rgb]{0.10,0.09,0.49}{{#1}}}
    \newcommand{\ControlFlowTok}[1]{\textcolor[rgb]{0.00,0.44,0.13}{\textbf{{#1}}}}
    \newcommand{\OperatorTok}[1]{\textcolor[rgb]{0.40,0.40,0.40}{{#1}}}
    \newcommand{\BuiltInTok}[1]{{#1}}
    \newcommand{\ExtensionTok}[1]{{#1}}
    \newcommand{\PreprocessorTok}[1]{\textcolor[rgb]{0.74,0.48,0.00}{{#1}}}
    \newcommand{\AttributeTok}[1]{\textcolor[rgb]{0.49,0.56,0.16}{{#1}}}
    \newcommand{\InformationTok}[1]{\textcolor[rgb]{0.38,0.63,0.69}{\textbf{\textit{{#1}}}}}
    \newcommand{\WarningTok}[1]{\textcolor[rgb]{0.38,0.63,0.69}{\textbf{\textit{{#1}}}}}


    % Define a nice break command that doesn't care if a line doesn't already
    % exist.
    \def\br{\hspace*{\fill} \\* }
    % Math Jax compatibility definitions
    \def\gt{>}
    \def\lt{<}
    \let\Oldtex\TeX
    \let\Oldlatex\LaTeX
    \renewcommand{\TeX}{\textrm{\Oldtex}}
    \renewcommand{\LaTeX}{\textrm{\Oldlatex}}
    % Document parameters
    % Document title
    \title{HW2}
    
    
    
    
    
    
    
% Pygments definitions
\makeatletter
\def\PY@reset{\let\PY@it=\relax \let\PY@bf=\relax%
    \let\PY@ul=\relax \let\PY@tc=\relax%
    \let\PY@bc=\relax \let\PY@ff=\relax}
\def\PY@tok#1{\csname PY@tok@#1\endcsname}
\def\PY@toks#1+{\ifx\relax#1\empty\else%
    \PY@tok{#1}\expandafter\PY@toks\fi}
\def\PY@do#1{\PY@bc{\PY@tc{\PY@ul{%
    \PY@it{\PY@bf{\PY@ff{#1}}}}}}}
\def\PY#1#2{\PY@reset\PY@toks#1+\relax+\PY@do{#2}}

\@namedef{PY@tok@w}{\def\PY@tc##1{\textcolor[rgb]{0.73,0.73,0.73}{##1}}}
\@namedef{PY@tok@c}{\let\PY@it=\textit\def\PY@tc##1{\textcolor[rgb]{0.24,0.48,0.48}{##1}}}
\@namedef{PY@tok@cp}{\def\PY@tc##1{\textcolor[rgb]{0.61,0.40,0.00}{##1}}}
\@namedef{PY@tok@k}{\let\PY@bf=\textbf\def\PY@tc##1{\textcolor[rgb]{0.00,0.50,0.00}{##1}}}
\@namedef{PY@tok@kp}{\def\PY@tc##1{\textcolor[rgb]{0.00,0.50,0.00}{##1}}}
\@namedef{PY@tok@kt}{\def\PY@tc##1{\textcolor[rgb]{0.69,0.00,0.25}{##1}}}
\@namedef{PY@tok@o}{\def\PY@tc##1{\textcolor[rgb]{0.40,0.40,0.40}{##1}}}
\@namedef{PY@tok@ow}{\let\PY@bf=\textbf\def\PY@tc##1{\textcolor[rgb]{0.67,0.13,1.00}{##1}}}
\@namedef{PY@tok@nb}{\def\PY@tc##1{\textcolor[rgb]{0.00,0.50,0.00}{##1}}}
\@namedef{PY@tok@nf}{\def\PY@tc##1{\textcolor[rgb]{0.00,0.00,1.00}{##1}}}
\@namedef{PY@tok@nc}{\let\PY@bf=\textbf\def\PY@tc##1{\textcolor[rgb]{0.00,0.00,1.00}{##1}}}
\@namedef{PY@tok@nn}{\let\PY@bf=\textbf\def\PY@tc##1{\textcolor[rgb]{0.00,0.00,1.00}{##1}}}
\@namedef{PY@tok@ne}{\let\PY@bf=\textbf\def\PY@tc##1{\textcolor[rgb]{0.80,0.25,0.22}{##1}}}
\@namedef{PY@tok@nv}{\def\PY@tc##1{\textcolor[rgb]{0.10,0.09,0.49}{##1}}}
\@namedef{PY@tok@no}{\def\PY@tc##1{\textcolor[rgb]{0.53,0.00,0.00}{##1}}}
\@namedef{PY@tok@nl}{\def\PY@tc##1{\textcolor[rgb]{0.46,0.46,0.00}{##1}}}
\@namedef{PY@tok@ni}{\let\PY@bf=\textbf\def\PY@tc##1{\textcolor[rgb]{0.44,0.44,0.44}{##1}}}
\@namedef{PY@tok@na}{\def\PY@tc##1{\textcolor[rgb]{0.41,0.47,0.13}{##1}}}
\@namedef{PY@tok@nt}{\let\PY@bf=\textbf\def\PY@tc##1{\textcolor[rgb]{0.00,0.50,0.00}{##1}}}
\@namedef{PY@tok@nd}{\def\PY@tc##1{\textcolor[rgb]{0.67,0.13,1.00}{##1}}}
\@namedef{PY@tok@s}{\def\PY@tc##1{\textcolor[rgb]{0.73,0.13,0.13}{##1}}}
\@namedef{PY@tok@sd}{\let\PY@it=\textit\def\PY@tc##1{\textcolor[rgb]{0.73,0.13,0.13}{##1}}}
\@namedef{PY@tok@si}{\let\PY@bf=\textbf\def\PY@tc##1{\textcolor[rgb]{0.64,0.35,0.47}{##1}}}
\@namedef{PY@tok@se}{\let\PY@bf=\textbf\def\PY@tc##1{\textcolor[rgb]{0.67,0.36,0.12}{##1}}}
\@namedef{PY@tok@sr}{\def\PY@tc##1{\textcolor[rgb]{0.64,0.35,0.47}{##1}}}
\@namedef{PY@tok@ss}{\def\PY@tc##1{\textcolor[rgb]{0.10,0.09,0.49}{##1}}}
\@namedef{PY@tok@sx}{\def\PY@tc##1{\textcolor[rgb]{0.00,0.50,0.00}{##1}}}
\@namedef{PY@tok@m}{\def\PY@tc##1{\textcolor[rgb]{0.40,0.40,0.40}{##1}}}
\@namedef{PY@tok@gh}{\let\PY@bf=\textbf\def\PY@tc##1{\textcolor[rgb]{0.00,0.00,0.50}{##1}}}
\@namedef{PY@tok@gu}{\let\PY@bf=\textbf\def\PY@tc##1{\textcolor[rgb]{0.50,0.00,0.50}{##1}}}
\@namedef{PY@tok@gd}{\def\PY@tc##1{\textcolor[rgb]{0.63,0.00,0.00}{##1}}}
\@namedef{PY@tok@gi}{\def\PY@tc##1{\textcolor[rgb]{0.00,0.52,0.00}{##1}}}
\@namedef{PY@tok@gr}{\def\PY@tc##1{\textcolor[rgb]{0.89,0.00,0.00}{##1}}}
\@namedef{PY@tok@ge}{\let\PY@it=\textit}
\@namedef{PY@tok@gs}{\let\PY@bf=\textbf}
\@namedef{PY@tok@ges}{\let\PY@bf=\textbf\let\PY@it=\textit}
\@namedef{PY@tok@gp}{\let\PY@bf=\textbf\def\PY@tc##1{\textcolor[rgb]{0.00,0.00,0.50}{##1}}}
\@namedef{PY@tok@go}{\def\PY@tc##1{\textcolor[rgb]{0.44,0.44,0.44}{##1}}}
\@namedef{PY@tok@gt}{\def\PY@tc##1{\textcolor[rgb]{0.00,0.27,0.87}{##1}}}
\@namedef{PY@tok@err}{\def\PY@bc##1{{\setlength{\fboxsep}{\string -\fboxrule}\fcolorbox[rgb]{1.00,0.00,0.00}{1,1,1}{\strut ##1}}}}
\@namedef{PY@tok@kc}{\let\PY@bf=\textbf\def\PY@tc##1{\textcolor[rgb]{0.00,0.50,0.00}{##1}}}
\@namedef{PY@tok@kd}{\let\PY@bf=\textbf\def\PY@tc##1{\textcolor[rgb]{0.00,0.50,0.00}{##1}}}
\@namedef{PY@tok@kn}{\let\PY@bf=\textbf\def\PY@tc##1{\textcolor[rgb]{0.00,0.50,0.00}{##1}}}
\@namedef{PY@tok@kr}{\let\PY@bf=\textbf\def\PY@tc##1{\textcolor[rgb]{0.00,0.50,0.00}{##1}}}
\@namedef{PY@tok@bp}{\def\PY@tc##1{\textcolor[rgb]{0.00,0.50,0.00}{##1}}}
\@namedef{PY@tok@fm}{\def\PY@tc##1{\textcolor[rgb]{0.00,0.00,1.00}{##1}}}
\@namedef{PY@tok@vc}{\def\PY@tc##1{\textcolor[rgb]{0.10,0.09,0.49}{##1}}}
\@namedef{PY@tok@vg}{\def\PY@tc##1{\textcolor[rgb]{0.10,0.09,0.49}{##1}}}
\@namedef{PY@tok@vi}{\def\PY@tc##1{\textcolor[rgb]{0.10,0.09,0.49}{##1}}}
\@namedef{PY@tok@vm}{\def\PY@tc##1{\textcolor[rgb]{0.10,0.09,0.49}{##1}}}
\@namedef{PY@tok@sa}{\def\PY@tc##1{\textcolor[rgb]{0.73,0.13,0.13}{##1}}}
\@namedef{PY@tok@sb}{\def\PY@tc##1{\textcolor[rgb]{0.73,0.13,0.13}{##1}}}
\@namedef{PY@tok@sc}{\def\PY@tc##1{\textcolor[rgb]{0.73,0.13,0.13}{##1}}}
\@namedef{PY@tok@dl}{\def\PY@tc##1{\textcolor[rgb]{0.73,0.13,0.13}{##1}}}
\@namedef{PY@tok@s2}{\def\PY@tc##1{\textcolor[rgb]{0.73,0.13,0.13}{##1}}}
\@namedef{PY@tok@sh}{\def\PY@tc##1{\textcolor[rgb]{0.73,0.13,0.13}{##1}}}
\@namedef{PY@tok@s1}{\def\PY@tc##1{\textcolor[rgb]{0.73,0.13,0.13}{##1}}}
\@namedef{PY@tok@mb}{\def\PY@tc##1{\textcolor[rgb]{0.40,0.40,0.40}{##1}}}
\@namedef{PY@tok@mf}{\def\PY@tc##1{\textcolor[rgb]{0.40,0.40,0.40}{##1}}}
\@namedef{PY@tok@mh}{\def\PY@tc##1{\textcolor[rgb]{0.40,0.40,0.40}{##1}}}
\@namedef{PY@tok@mi}{\def\PY@tc##1{\textcolor[rgb]{0.40,0.40,0.40}{##1}}}
\@namedef{PY@tok@il}{\def\PY@tc##1{\textcolor[rgb]{0.40,0.40,0.40}{##1}}}
\@namedef{PY@tok@mo}{\def\PY@tc##1{\textcolor[rgb]{0.40,0.40,0.40}{##1}}}
\@namedef{PY@tok@ch}{\let\PY@it=\textit\def\PY@tc##1{\textcolor[rgb]{0.24,0.48,0.48}{##1}}}
\@namedef{PY@tok@cm}{\let\PY@it=\textit\def\PY@tc##1{\textcolor[rgb]{0.24,0.48,0.48}{##1}}}
\@namedef{PY@tok@cpf}{\let\PY@it=\textit\def\PY@tc##1{\textcolor[rgb]{0.24,0.48,0.48}{##1}}}
\@namedef{PY@tok@c1}{\let\PY@it=\textit\def\PY@tc##1{\textcolor[rgb]{0.24,0.48,0.48}{##1}}}
\@namedef{PY@tok@cs}{\let\PY@it=\textit\def\PY@tc##1{\textcolor[rgb]{0.24,0.48,0.48}{##1}}}

\def\PYZbs{\char`\\}
\def\PYZus{\char`\_}
\def\PYZob{\char`\{}
\def\PYZcb{\char`\}}
\def\PYZca{\char`\^}
\def\PYZam{\char`\&}
\def\PYZlt{\char`\<}
\def\PYZgt{\char`\>}
\def\PYZsh{\char`\#}
\def\PYZpc{\char`\%}
\def\PYZdl{\char`\$}
\def\PYZhy{\char`\-}
\def\PYZsq{\char`\'}
\def\PYZdq{\char`\"}
\def\PYZti{\char`\~}
% for compatibility with earlier versions
\def\PYZat{@}
\def\PYZlb{[}
\def\PYZrb{]}
\makeatother


    % For linebreaks inside Verbatim environment from package fancyvrb.
    \makeatletter
        \newbox\Wrappedcontinuationbox
        \newbox\Wrappedvisiblespacebox
        \newcommand*\Wrappedvisiblespace {\textcolor{red}{\textvisiblespace}}
        \newcommand*\Wrappedcontinuationsymbol {\textcolor{red}{\llap{\tiny$\m@th\hookrightarrow$}}}
        \newcommand*\Wrappedcontinuationindent {3ex }
        \newcommand*\Wrappedafterbreak {\kern\Wrappedcontinuationindent\copy\Wrappedcontinuationbox}
        % Take advantage of the already applied Pygments mark-up to insert
        % potential linebreaks for TeX processing.
        %        {, <, #, %, $, ' and ": go to next line.
        %        _, }, ^, &, >, - and ~: stay at end of broken line.
        % Use of \textquotesingle for straight quote.
        \newcommand*\Wrappedbreaksatspecials {%
            \def\PYGZus{\discretionary{\char`\_}{\Wrappedafterbreak}{\char`\_}}%
            \def\PYGZob{\discretionary{}{\Wrappedafterbreak\char`\{}{\char`\{}}%
            \def\PYGZcb{\discretionary{\char`\}}{\Wrappedafterbreak}{\char`\}}}%
            \def\PYGZca{\discretionary{\char`\^}{\Wrappedafterbreak}{\char`\^}}%
            \def\PYGZam{\discretionary{\char`\&}{\Wrappedafterbreak}{\char`\&}}%
            \def\PYGZlt{\discretionary{}{\Wrappedafterbreak\char`\<}{\char`\<}}%
            \def\PYGZgt{\discretionary{\char`\>}{\Wrappedafterbreak}{\char`\>}}%
            \def\PYGZsh{\discretionary{}{\Wrappedafterbreak\char`\#}{\char`\#}}%
            \def\PYGZpc{\discretionary{}{\Wrappedafterbreak\char`\%}{\char`\%}}%
            \def\PYGZdl{\discretionary{}{\Wrappedafterbreak\char`\$}{\char`\$}}%
            \def\PYGZhy{\discretionary{\char`\-}{\Wrappedafterbreak}{\char`\-}}%
            \def\PYGZsq{\discretionary{}{\Wrappedafterbreak\textquotesingle}{\textquotesingle}}%
            \def\PYGZdq{\discretionary{}{\Wrappedafterbreak\char`\"}{\char`\"}}%
            \def\PYGZti{\discretionary{\char`\~}{\Wrappedafterbreak}{\char`\~}}%
        }
        % Some characters . , ; ? ! / are not pygmentized.
        % This macro makes them "active" and they will insert potential linebreaks
        \newcommand*\Wrappedbreaksatpunct {%
            \lccode`\~`\.\lowercase{\def~}{\discretionary{\hbox{\char`\.}}{\Wrappedafterbreak}{\hbox{\char`\.}}}%
            \lccode`\~`\,\lowercase{\def~}{\discretionary{\hbox{\char`\,}}{\Wrappedafterbreak}{\hbox{\char`\,}}}%
            \lccode`\~`\;\lowercase{\def~}{\discretionary{\hbox{\char`\;}}{\Wrappedafterbreak}{\hbox{\char`\;}}}%
            \lccode`\~`\:\lowercase{\def~}{\discretionary{\hbox{\char`\:}}{\Wrappedafterbreak}{\hbox{\char`\:}}}%
            \lccode`\~`\?\lowercase{\def~}{\discretionary{\hbox{\char`\?}}{\Wrappedafterbreak}{\hbox{\char`\?}}}%
            \lccode`\~`\!\lowercase{\def~}{\discretionary{\hbox{\char`\!}}{\Wrappedafterbreak}{\hbox{\char`\!}}}%
            \lccode`\~`\/\lowercase{\def~}{\discretionary{\hbox{\char`\/}}{\Wrappedafterbreak}{\hbox{\char`\/}}}%
            \catcode`\.\active
            \catcode`\,\active
            \catcode`\;\active
            \catcode`\:\active
            \catcode`\?\active
            \catcode`\!\active
            \catcode`\/\active
            \lccode`\~`\~
        }
    \makeatother

    \let\OriginalVerbatim=\Verbatim
    \makeatletter
    \renewcommand{\Verbatim}[1][1]{%
        %\parskip\z@skip
        \sbox\Wrappedcontinuationbox {\Wrappedcontinuationsymbol}%
        \sbox\Wrappedvisiblespacebox {\FV@SetupFont\Wrappedvisiblespace}%
        \def\FancyVerbFormatLine ##1{\hsize\linewidth
            \vtop{\raggedright\hyphenpenalty\z@\exhyphenpenalty\z@
                \doublehyphendemerits\z@\finalhyphendemerits\z@
                \strut ##1\strut}%
        }%
        % If the linebreak is at a space, the latter will be displayed as visible
        % space at end of first line, and a continuation symbol starts next line.
        % Stretch/shrink are however usually zero for typewriter font.
        \def\FV@Space {%
            \nobreak\hskip\z@ plus\fontdimen3\font minus\fontdimen4\font
            \discretionary{\copy\Wrappedvisiblespacebox}{\Wrappedafterbreak}
            {\kern\fontdimen2\font}%
        }%

        % Allow breaks at special characters using \PYG... macros.
        \Wrappedbreaksatspecials
        % Breaks at punctuation characters . , ; ? ! and / need catcode=\active
        \OriginalVerbatim[#1,codes*=\Wrappedbreaksatpunct]%
    }
    \makeatother

    % Exact colors from NB
    \definecolor{incolor}{HTML}{303F9F}
    \definecolor{outcolor}{HTML}{D84315}
    \definecolor{cellborder}{HTML}{CFCFCF}
    \definecolor{cellbackground}{HTML}{F7F7F7}

    % prompt
    \makeatletter
    \newcommand{\boxspacing}{\kern\kvtcb@left@rule\kern\kvtcb@boxsep}
    \makeatother
    \newcommand{\prompt}[4]{
        {\ttfamily\llap{{\color{#2}[#3]:\hspace{3pt}#4}}\vspace{-\baselineskip}}
    }
    

    
    % Prevent overflowing lines due to hard-to-break entities
    \sloppy
    % Setup hyperref package
    \hypersetup{
      breaklinks=true,  % so long urls are correctly broken across lines
      colorlinks=true,
      urlcolor=urlcolor,
      linkcolor=linkcolor,
      citecolor=citecolor,
      }
    % Slightly bigger margins than the latex defaults
    
    \geometry{verbose,tmargin=1in,bmargin=1in,lmargin=1in,rmargin=1in}
    
    

\begin{document}
    
    \maketitle
    
    

    
    \hypertarget{homework-assignment-2}{%
\section{Homework Assignment 2}\label{homework-assignment-2}}

    \begin{tcolorbox}[breakable, size=fbox, boxrule=1pt, pad at break*=1mm,colback=cellbackground, colframe=cellborder]
\prompt{In}{incolor}{4}{\boxspacing}
\begin{Verbatim}[commandchars=\\\{\}]
\PY{k+kn}{from} \PY{n+nn}{sympy} \PY{k+kn}{import}\PY{o}{*}
\PY{k+kn}{import} \PY{n+nn}{numpy} \PY{k}{as} \PY{n+nn}{np}
\PY{k+kn}{from} \PY{n+nn}{sympy}\PY{n+nn}{.}\PY{n+nn}{abc} \PY{k+kn}{import} \PY{n}{x}\PY{p}{,}\PY{n}{y}\PY{p}{,}\PY{n}{z}\PY{p}{,}\PY{n}{u}\PY{p}{,}\PY{n}{v}\PY{p}{,}\PY{n}{w}\PY{p}{,}\PY{n}{t}
\PY{k+kn}{from} \PY{n+nn}{dtumathtools} \PY{k+kn}{import} \PY{o}{*}
\PY{k+kn}{from} \PY{n+nn}{IPython}\PY{n+nn}{.}\PY{n+nn}{display} \PY{k+kn}{import} \PY{n}{display}\PY{p}{,} \PY{n}{Latex}\PY{p}{,} \PY{n}{Image}
\end{Verbatim}
\end{tcolorbox}

    \hypertarget{problem-1}{%
\subsection{Problem 1}\label{problem-1}}

    A function \(f:\mathbb{R} \rightarrow \mathbb{R}\) is given by the
expression:

\begin{equation*}
f(x)=\cos (3x).
\end{equation*}

\begin{enumerate}
\def\labelenumi{\alph{enumi})}
\item
  Determine the approximating polynomial \(P_{4,f,x_0}\) of degree (at
  most) \(4\) with expansion point \(x_0 = 0\).
\item
  Use the polynomial you found to compute an approximate value of
  \(\cos (\frac{1}{2} )\), and estimate how far the found value is from
  the exact value.
\end{enumerate}

We are given the following information about a function \(g\):

\begin{equation*}
g(0)=1  \quad \text{and} \quad  g^{(n)} (0)=  
\begin{cases}  
  (-1)^{\frac{n}{2}}2^n & n \text{ is even}, \\
  0 &  n  \text{ is odd}.
\end{cases}  
\end{equation*}

\begin{enumerate}
\def\labelenumi{\alph{enumi})}
\setcounter{enumi}{2}
\tightlist
\item
  Determine the approximating polynomial \(P_{7,g,x_0}\) of \(g\) of
  degree (at most) \(7\) with expansion point \(x_0 = 0\). Give your
  best suggestion on which function \(g\) might be.
\end{enumerate}

    \begin{tcolorbox}[breakable, size=fbox, boxrule=1pt, pad at break*=1mm,colback=cellbackground, colframe=cellborder]
\prompt{In}{incolor}{4}{\boxspacing}
\begin{Verbatim}[commandchars=\\\{\}]
\PY{n}{f} \PY{o}{=} \PY{n}{cos}\PY{p}{(}\PY{l+m+mi}{3}\PY{o}{*}\PY{n}{x}\PY{p}{)}  \PY{c+c1}{\PYZsh{}define the function}
\PY{n}{P\PYZus{}4} \PY{o}{=} \PY{n}{series}\PY{p}{(}\PY{n}{f}\PY{p}{,}\PY{n}{x0}\PY{o}{=}\PY{l+m+mi}{0}\PY{p}{,}\PY{n}{n}\PY{o}{=}\PY{l+m+mi}{5}\PY{p}{)}\PY{o}{.}\PY{n}{removeO}\PY{p}{(}\PY{p}{)}  \PY{c+c1}{\PYZsh{}determining the approximation polynomial of degree 4}
\PY{n}{P\PYZus{}4}
\end{Verbatim}
\end{tcolorbox}
 
            
\prompt{Out}{outcolor}{4}{}
    
    $\displaystyle \frac{27 x^{4}}{8} - \frac{9 x^{2}}{2} + 1$

    

    To evaluate \(\cos(\frac{1}{2})\) we need to find such \(x\) that
\(3x = \frac{1}{2}\), \(x = \frac{1}{6}\).

    \begin{tcolorbox}[breakable, size=fbox, boxrule=1pt, pad at break*=1mm,colback=cellbackground, colframe=cellborder]
\prompt{In}{incolor}{54}{\boxspacing}
\begin{Verbatim}[commandchars=\\\{\}]
\PY{n}{display}\PY{p}{(}\PY{n}{Latex}\PY{p}{(}\PY{l+s+sa}{fr}\PY{l+s+s1}{\PYZsq{}}\PY{l+s+s1}{\PYZdl{}}\PY{l+s+s1}{\PYZbs{}}\PY{l+s+s1}{cos (}\PY{l+s+s1}{\PYZbs{}}\PY{l+s+s1}{frac}\PY{l+s+si}{\PYZob{}}\PY{l+m+mi}{1}\PY{l+s+si}{\PYZcb{}}\PY{l+s+si}{\PYZob{}}\PY{l+m+mi}{2}\PY{l+s+si}{\PYZcb{}}\PY{l+s+s1}{ ) = \PYZdl{} }\PY{l+s+si}{\PYZob{}}\PY{n}{P\PYZus{}4}\PY{o}{.}\PY{n}{subs}\PY{p}{(}\PY{n}{x}\PY{p}{,}\PY{l+m+mi}{1}\PY{o}{/}\PY{l+m+mi}{6}\PY{p}{)}\PY{l+s+si}{\PYZcb{}}\PY{l+s+s1}{\PYZsq{}}\PY{p}{)}\PY{p}{)} \PY{c+c1}{\PYZsh{}substitute x = 1/6 into P\PYZus{}4 expression and display it}
\end{Verbatim}
\end{tcolorbox}

    $\cos (\frac12 ) = $ 0.877604166666667

    
    The error estimation is then
\(R_{n,f,x_0} = f^{(n+1)}_{\xi}\frac{1}{(n+1)!}(x-x_0)^{n+1}\), where
\(\xi \in[x_0,x], n=4\). \begin{equation}
R_{4,f,0}(x) = f^{(5)}_{\xi}\frac{1}{5!}(x)^{5}
\end{equation} \begin{equation}
f^{(5)}_{\xi} = -27*9\sin(3\xi), \xi \in [0, \frac{1}{6}]
\end{equation} We need to choose \(\xi\) such that \(|R_{4,f,0}|\) is
the maximum possible value on the \([0, \frac{1}{6}]\) interval,
therefore \(\xi = \frac{1}{6}\). \begin{equation}
R_{4,f,0}(\frac{1}{6}) = \frac{-27*9}{5!}\sin(3\frac{1}{6})(\frac{1}{6})^{5}
\end{equation}

    \begin{tcolorbox}[breakable, size=fbox, boxrule=1pt, pad at break*=1mm,colback=cellbackground, colframe=cellborder]
\prompt{In}{incolor}{59}{\boxspacing}
\begin{Verbatim}[commandchars=\\\{\}]
\PY{n}{R\PYZus{}4} \PY{o}{=} \PY{o}{\PYZhy{}}\PY{l+m+mi}{27}\PY{o}{*}\PY{l+m+mi}{9}\PY{o}{/}\PY{p}{(}\PY{l+m+mi}{120} \PY{o}{*} \PY{l+m+mi}{6}\PY{o}{*}\PY{o}{*}\PY{l+m+mi}{5}\PY{p}{)}\PY{o}{*}\PY{n}{sin}\PY{p}{(}\PY{l+m+mf}{0.5}\PY{p}{)}
\PY{n+nb}{abs}\PY{p}{(}\PY{n}{R\PYZus{}4}\PY{p}{)}
\end{Verbatim}
\end{tcolorbox}
 
            
\prompt{Out}{outcolor}{59}{}
    
    $\displaystyle 0.000124850400678178$

    

    The greates error \(R_{4,f,0}(\frac{1}{6}) = 0.0001248\)

    \begin{tcolorbox}[breakable, size=fbox, boxrule=1pt, pad at break*=1mm,colback=cellbackground, colframe=cellborder]
\prompt{In}{incolor}{62}{\boxspacing}
\begin{Verbatim}[commandchars=\\\{\}]
\PY{n}{display}\PY{p}{(}\PY{n}{Latex}\PY{p}{(}\PY{l+s+sa}{fr}\PY{l+s+s1}{\PYZsq{}}\PY{l+s+s1}{\PYZdl{}}\PY{l+s+s1}{\PYZbs{}}\PY{l+s+s1}{cos(}\PY{l+s+s1}{\PYZbs{}}\PY{l+s+s1}{frac}\PY{l+s+si}{\PYZob{}}\PY{l+m+mi}{1}\PY{l+s+si}{\PYZcb{}}\PY{l+s+si}{\PYZob{}}\PY{l+m+mi}{2}\PY{l+s+si}{\PYZcb{}}\PY{l+s+s1}{) }\PY{l+s+s1}{\PYZbs{}}\PY{l+s+s1}{in\PYZdl{} [}\PY{l+s+si}{\PYZob{}}\PY{n}{P\PYZus{}4}\PY{o}{.}\PY{n}{subs}\PY{p}{(}\PY{n}{x}\PY{p}{,}\PY{l+m+mi}{1}\PY{o}{/}\PY{l+m+mi}{6}\PY{p}{)}\PY{o}{+}\PY{n}{R\PYZus{}4}\PY{l+s+si}{\PYZcb{}}\PY{l+s+s1}{, }\PY{l+s+si}{\PYZob{}}\PY{n}{P\PYZus{}4}\PY{o}{.}\PY{n}{subs}\PY{p}{(}\PY{n}{x}\PY{p}{,}\PY{l+m+mi}{1}\PY{o}{/}\PY{l+m+mi}{6}\PY{p}{)}\PY{o}{\PYZhy{}}\PY{n}{R\PYZus{}4}\PY{l+s+si}{\PYZcb{}}\PY{l+s+s1}{]}\PY{l+s+s1}{\PYZsq{}}\PY{p}{)}\PY{p}{)}
\end{Verbatim}
\end{tcolorbox}

    $\cos(\frac12) \in$ [0.877479316265988, 0.877729017067345]

    
    The true value of \(\cos(\frac{1}{2})\) is \(0.8775826\), which is in
the found range above.

    We are given the following information about a function \(g\):

\begin{equation*}
g(0)=1  \quad \text{and} \quad  g^{(n)} (0)=  
\begin{cases}  
  (-1)^{\frac{n}{2}}2^n & n \text{ is even}, \\
  0 &  n  \text{ is odd}.
\end{cases}  
\end{equation*} The approximating polynomial \(P_{7,g,x_0}\) of \(g\) of
degree \(7\) with expansion point \(x_0 = 0\) is then will be
\begin{align}
& P_{7,g,0}(x) = g(0) + g^{1}(0)x +  \frac{1}{2}g^{2}(0)x^2 + \frac{1}{6}g^{3}(0)x^3 + \frac{1}{24}g^{4}(0)x^4 + \frac{1}{120}g^{5}(0)x^5 + \frac{1}{720}g^{6}(0)x^6 + \frac{1}{5040}g^{7}(0)x^7 \\
& P_{7,g,0}(x) = 1 + 0*x - 2x^2 + \frac{1}{6}*0*x^3 + \frac{2}{3}x^4 + \frac{1}{120}*0*x^5 - \frac{4}{45}g^{6}(0)x^6 + \frac{1}{5040}*0*x^7 \\
& P_{7,g,0}(x) = 1 - 2x^2 + \frac{2}{3}x^4 - \frac{4}{45}x^6
\end{align}

    The expansion of the function \(g\) we got above is very similar to the
expansion of the \(f\) function, so the \(g\) function can also be some
sort of \(\cos\). If we take a look at the \(-2x^2\) term, which equals
\(g^{(2)}(0)\), and assuming \(g(x) = \cos(ax)\), the \(a\) should be
\(2\).

We can check it by plotting the expansion and the function on the
interval \([x_1,x_2]\) near the point \(x_0\).

    \begin{tcolorbox}[breakable, size=fbox, boxrule=1pt, pad at break*=1mm,colback=cellbackground, colframe=cellborder]
\prompt{In}{incolor}{82}{\boxspacing}
\begin{Verbatim}[commandchars=\\\{\}]
\PY{n}{P\PYZus{}7} \PY{o}{=} \PY{l+m+mi}{1}\PY{o}{\PYZhy{}}\PY{l+m+mi}{2}\PY{o}{*}\PY{n}{x}\PY{o}{*}\PY{o}{*}\PY{l+m+mi}{2} \PY{o}{+} \PY{l+m+mi}{2}\PY{o}{/}\PY{l+m+mi}{3}\PY{o}{*}\PY{n}{x}\PY{o}{*}\PY{o}{*}\PY{l+m+mi}{4} \PY{o}{\PYZhy{}} \PY{l+m+mi}{4}\PY{o}{/}\PY{l+m+mi}{45}\PY{o}{*}\PY{n}{x}\PY{o}{*}\PY{o}{*}\PY{l+m+mi}{6}
\PY{n}{fig} \PY{o}{=} \PY{n}{dtuplot}\PY{o}{.}\PY{n}{plot}\PY{p}{(}\PY{n}{P\PYZus{}7}\PY{p}{,}\PY{p}{(}\PY{n}{x}\PY{p}{,} \PY{o}{\PYZhy{}}\PY{l+m+mi}{2}\PY{p}{,}\PY{l+m+mi}{2}\PY{p}{)}\PY{p}{,} \PY{n}{show}\PY{o}{=}\PY{k+kc}{False}\PY{p}{)}
\PY{c+c1}{\PYZsh{}fig.extend(dtuplot.plot(P\PYZus{}4,(x, \PYZhy{}2,2), show=False))}
\PY{n}{fig}\PY{o}{.}\PY{n}{extend}\PY{p}{(}\PY{n}{dtuplot}\PY{o}{.}\PY{n}{plot}\PY{p}{(}\PY{n}{cos}\PY{p}{(}\PY{l+m+mi}{2}\PY{o}{*}\PY{n}{x}\PY{p}{)}\PY{p}{,}\PY{p}{(}\PY{n}{x}\PY{p}{,}\PY{o}{\PYZhy{}}\PY{l+m+mi}{2}\PY{p}{,}\PY{l+m+mi}{2}\PY{p}{)}\PY{p}{,}\PY{n}{show}\PY{o}{=}\PY{k+kc}{False}\PY{p}{)}\PY{p}{)}
\PY{n}{fig}\PY{o}{.}\PY{n}{show}\PY{p}{(}\PY{p}{)}
\end{Verbatim}
\end{tcolorbox}

    \begin{center}
    \adjustimage{max size={0.9\linewidth}{0.9\paperheight}}{output_14_0.png}
    \end{center}
    { \hspace*{\fill} \\}
    
    Based on the graph, our assumption can be considered correct.
\(g(x) = cos(2x)\)

    \hypertarget{problem-2}{%
\subsection{Problem 2}\label{problem-2}}

    Let \(B\) be the set
\(B=\left\{ (x_1 , x_2) \in \mathbb{R}^2 \mid x_1^2 + x_2^2 \leq 1 \wedge x_1 \geq 0 \right\}\).

A function \(f:B \rightarrow \mathbb{R}\) is given by

\begin{equation*}
f(x_1 ,x_2 )=x_1^2 x_2^2 +3x_1 x_2 +x_2 -4.
\end{equation*}

State the image of \(f\).

    To find the image of \(f\) we need to find the local maximum and minimum
of the function among the set of points \((x1, x2) \in B\).

The first step will be defining the parameterization plane. We know that
\(x_1^2 + x_2^2 \leq 1 \wedge x_1 \geq 0\) is a half circle.
\begin{equation}
r = [t \ cos(u), t \sin(u)], t \in [0,1], u \in [-\frac{\pi}{2}, \frac{\pi}{2}]
\end{equation}

To check the parameterization plane we can plot it

    \begin{tcolorbox}[breakable, size=fbox, boxrule=1pt, pad at break*=1mm,colback=cellbackground, colframe=cellborder]
\prompt{In}{incolor}{6}{\boxspacing}
\begin{Verbatim}[commandchars=\\\{\}]
\PY{n}{r} \PY{o}{=} \PY{p}{[}\PY{n}{t}\PY{o}{*}\PY{n}{cos}\PY{p}{(}\PY{n}{u}\PY{p}{)}\PY{p}{,} \PY{n}{t}\PY{o}{*}\PY{n}{sin}\PY{p}{(}\PY{n}{u}\PY{p}{)}\PY{p}{]} \PY{c+c1}{\PYZsh{} defining a parameterization plane}
\PY{n}{f} \PY{o}{=} \PY{n}{x}\PY{o}{*}\PY{o}{*}\PY{l+m+mi}{2}\PY{o}{*}\PY{n}{y}\PY{o}{*}\PY{o}{*}\PY{l+m+mi}{2} \PY{o}{+} \PY{l+m+mi}{3}\PY{o}{*}\PY{n}{x}\PY{o}{*}\PY{n}{y} \PY{o}{+} \PY{n}{y} \PY{o}{\PYZhy{}} \PY{l+m+mi}{4}  \PY{c+c1}{\PYZsh{} defining a function f with x1=x and x2=y}
\PY{n}{param} \PY{o}{=} \PY{p}{\PYZob{}}\PY{n}{x}\PY{p}{:} \PY{n}{r}\PY{p}{[}\PY{l+m+mi}{0}\PY{p}{]}\PY{p}{,} \PY{n}{y}\PY{p}{:} \PY{n}{r}\PY{p}{[}\PY{l+m+mi}{1}\PY{p}{]}\PY{p}{\PYZcb{}}   \PY{c+c1}{\PYZsh{} creating  a dictionary of parametric expressions}
\PY{n}{dtuplot}\PY{o}{.}\PY{n}{plot3d\PYZus{}parametric\PYZus{}surface}\PY{p}{(}\PY{n}{r}\PY{p}{[}\PY{l+m+mi}{0}\PY{p}{]}\PY{p}{,}\PY{n}{r}\PY{p}{[}\PY{l+m+mi}{1}\PY{p}{]}\PY{p}{,} \PY{l+m+mi}{0}\PY{p}{,} \PY{p}{(}\PY{n}{u}\PY{p}{,}\PY{o}{\PYZhy{}}\PY{n}{pi}\PY{o}{/}\PY{l+m+mi}{2}\PY{p}{,}\PY{n}{pi}\PY{o}{/}\PY{l+m+mi}{2}\PY{p}{)}\PY{p}{,} \PY{p}{(}\PY{n}{t}\PY{p}{,} \PY{l+m+mi}{0}\PY{p}{,} \PY{l+m+mi}{1}\PY{p}{)}\PY{p}{)}  \PY{c+c1}{\PYZsh{} plotting the plane }
\PY{n}{dtuplot}\PY{o}{.}\PY{n}{plot3d}\PY{p}{(}\PY{n}{f}\PY{o}{.}\PY{n}{subs}\PY{p}{(}\PY{n}{param}\PY{p}{)}\PY{p}{,} \PY{p}{(}\PY{n}{u}\PY{p}{,}\PY{o}{\PYZhy{}}\PY{n}{pi}\PY{o}{/}\PY{l+m+mi}{2}\PY{p}{,}\PY{n}{pi}\PY{o}{/}\PY{l+m+mi}{2}\PY{p}{)}\PY{p}{,} \PY{p}{(}\PY{n}{t}\PY{p}{,} \PY{l+m+mi}{0}\PY{p}{,} \PY{l+m+mi}{1}\PY{p}{)}\PY{p}{)}  \PY{c+c1}{\PYZsh{} plotting the function parameterized by our plane }
\PY{c+c1}{\PYZsh{}dtuplot.plot3d(f, (x, \PYZhy{}5, 5), (y, \PYZhy{}5, 5))  \PYZsh{} function itself}
\end{Verbatim}
\end{tcolorbox}

    \begin{center}
    \adjustimage{max size={0.9\linewidth}{0.9\paperheight}}{output_19_0.png}
    \end{center}
    { \hspace*{\fill} \\}
    
    \begin{center}
    \adjustimage{max size={0.9\linewidth}{0.9\paperheight}}{output_19_1.png}
    \end{center}
    { \hspace*{\fill} \\}
    
            \begin{tcolorbox}[breakable, size=fbox, boxrule=.5pt, pad at break*=1mm, opacityfill=0]
\prompt{Out}{outcolor}{6}{\boxspacing}
\begin{Verbatim}[commandchars=\\\{\}]
<spb.backends.matplotlib.matplotlib.MatplotlibBackend at 0x76013d7e7f70>
\end{Verbatim}
\end{tcolorbox}
        
    From the 2 plots above it is evident that \(f\) is continuous and
differentiable for all
\(t \in [0,1] \text{ and } u \in [-\frac{\pi}{2}, \frac{\pi}{2}]\).

The next step will be finding stationary points by finding the gradient
of the function and equating it to 0.

    \begin{tcolorbox}[breakable, size=fbox, boxrule=1pt, pad at break*=1mm,colback=cellbackground, colframe=cellborder]
\prompt{In}{incolor}{160}{\boxspacing}
\begin{Verbatim}[commandchars=\\\{\}]
\PY{n}{Nf} \PY{o}{=} \PY{n}{dtutools}\PY{o}{.}\PY{n}{gradient}\PY{p}{(}\PY{n}{f}\PY{p}{)} \PY{c+c1}{\PYZsh{} finding gradient}
\PY{n}{S} \PY{o}{=} \PY{n}{Nf}\PY{o}{.}\PY{n}{subs}\PY{p}{(}\PY{n}{param}\PY{p}{)} \PY{c+c1}{\PYZsh{} substituting t*cos(u) instead of x and t*sin(u) instead of y}
\PY{n}{sol} \PY{o}{=} \PY{n}{solve}\PY{p}{(}\PY{n}{S}\PY{p}{,} \PY{n}{u}\PY{p}{,} \PY{n}{t}\PY{p}{)} \PY{c+c1}{\PYZsh{} solving for t and u}
\PY{n}{sol\PYZus{}check} \PY{o}{=} \PY{n}{solve}\PY{p}{(}\PY{n}{Nf}\PY{p}{,} \PY{n}{x}\PY{p}{,} \PY{n}{y}\PY{p}{)} \PY{c+c1}{\PYZsh{} the same thing without substituting parameterization to check}
\PY{n}{sol}\PY{p}{,} \PY{n}{sol\PYZus{}check}  \PY{c+c1}{\PYZsh{} displaying solutions}
\end{Verbatim}
\end{tcolorbox}

            \begin{tcolorbox}[breakable, size=fbox, boxrule=.5pt, pad at break*=1mm, opacityfill=0]
\prompt{Out}{outcolor}{160}{\boxspacing}
\begin{Verbatim}[commandchars=\\\{\}]
([(0, -1/3)], [(-1/3, 0)])
\end{Verbatim}
\end{tcolorbox}
        
    \(u = 0\) and \(t = -\frac{1}{3}\), as we can see the \(t\) value is out
of the range we defined previously, and it doesn't make any sense for
the radius to be negative, therefore there are no stationary points.

The third and last step would be boundary investigation, as it is
obvious from the plot above that the function has a global maximum and
minimum on the boundary.

As it is the boundary of \$ r(u,t) = {[}t ~cos(u), t \sin(u){]}, t
\in [0,1], u \in [-\frac{\pi}{2}, \frac{\pi}{2}]\$, \(t\) must be \(1\).
We should have also tested the boundary
\(r_{1}(u) = [0, u], u \in [0,1]\), but as we have plotted the function
and it is clear that maximum and minimum values are on the boundary
parameterized by the \(r(u,1) = [\cos(u),\sin(u)]\) we won't further
investigate the \(r_{1}(u)\) boundary.

    \begin{tcolorbox}[breakable, size=fbox, boxrule=1pt, pad at break*=1mm,colback=cellbackground, colframe=cellborder]
\prompt{In}{incolor}{10}{\boxspacing}
\begin{Verbatim}[commandchars=\\\{\}]
\PY{n}{param1} \PY{o}{=} \PY{p}{\PYZob{}}\PY{n}{x}\PY{p}{:} \PY{n}{cos}\PY{p}{(}\PY{n}{u}\PY{p}{)}\PY{p}{,} \PY{n}{y}\PY{p}{:} \PY{n}{sin}\PY{p}{(}\PY{n}{u}\PY{p}{)}\PY{p}{\PYZcb{}} \PY{c+c1}{\PYZsh{} creating  a dictionary of parametric expressions}
\PY{n}{dtuplot}\PY{o}{.}\PY{n}{plot}\PY{p}{(}\PY{n}{f}\PY{o}{.}\PY{n}{subs}\PY{p}{(}\PY{n}{param1}\PY{p}{)}\PY{p}{,} \PY{p}{(}\PY{n}{u}\PY{p}{,}\PY{o}{\PYZhy{}}\PY{n}{pi}\PY{o}{/}\PY{l+m+mi}{2}\PY{p}{,}\PY{n}{pi}\PY{o}{/}\PY{l+m+mi}{2}\PY{p}{)}\PY{p}{)} \PY{c+c1}{\PYZsh{} plotting the line we got for u in [\PYZhy{}pi/2, pi/2]}
\PY{n}{B} \PY{o}{=} \PY{n}{dtutools}\PY{o}{.}\PY{n}{gradient}\PY{p}{(}\PY{n}{f}\PY{o}{.}\PY{n}{subs}\PY{p}{(}\PY{n}{param1}\PY{p}{)}\PY{p}{)} \PY{c+c1}{\PYZsh{} finding gradient(derivative) with respect to u of the function f}
\PY{n}{display}\PY{p}{(}\PY{n}{B}\PY{p}{)} \PY{c+c1}{\PYZsh{}display the derivative}
\end{Verbatim}
\end{tcolorbox}

    \begin{center}
    \adjustimage{max size={0.9\linewidth}{0.9\paperheight}}{output_23_0.png}
    \end{center}
    { \hspace*{\fill} \\}
    
    $\displaystyle - 2 \sin^{3}{\left(u \right)} \cos{\left(u \right)} - 3 \sin^{2}{\left(u \right)} + 2 \sin{\left(u \right)} \cos^{3}{\left(u \right)} + 3 \cos^{2}{\left(u \right)} + \cos{\left(u \right)}$

    
    Before we proceed to solve the equation, it is worth mentioning that
solving the equation above symbolically, e.g.~using \texttt{solve()},
yields a system of equations. We would rather use numerical solver
\texttt{nsolve}.

    \begin{tcolorbox}[breakable, size=fbox, boxrule=1pt, pad at break*=1mm,colback=cellbackground, colframe=cellborder]
\prompt{In}{incolor}{164}{\boxspacing}
\begin{Verbatim}[commandchars=\\\{\}]
\PY{n}{sol\PYZus{}1} \PY{o}{=} \PY{n}{nsolve}\PY{p}{(}\PY{n}{B}\PY{p}{,} \PY{n}{u}\PY{p}{,} \PY{o}{\PYZhy{}}\PY{l+m+mi}{1}\PY{p}{)} \PY{c+c1}{\PYZsh{} find first solution around u=\PYZhy{}1}
\PY{n}{sol\PYZus{}2} \PY{o}{=} \PY{n}{nsolve}\PY{p}{(}\PY{n}{B}\PY{p}{,} \PY{n}{u}\PY{p}{,} \PY{l+m+mf}{0.8}\PY{p}{)} \PY{c+c1}{\PYZsh{} finding second solution around u=0.8}
\PY{n}{sol\PYZus{}1}\PY{p}{,} \PY{n}{sol\PYZus{}2}
\end{Verbatim}
\end{tcolorbox}

            \begin{tcolorbox}[breakable, size=fbox, boxrule=.5pt, pad at break*=1mm, opacityfill=0]
\prompt{Out}{outcolor}{164}{\boxspacing}
\begin{Verbatim}[commandchars=\\\{\}]
(-0.933199076006191, 0.866924817752369)
\end{Verbatim}
\end{tcolorbox}
        
    \begin{tcolorbox}[breakable, size=fbox, boxrule=1pt, pad at break*=1mm,colback=cellbackground, colframe=cellborder]
\prompt{In}{incolor}{167}{\boxspacing}
\begin{Verbatim}[commandchars=\\\{\}]
\PY{n}{f}\PY{o}{.}\PY{n}{subs}\PY{p}{(}\PY{p}{\PYZob{}}\PY{n}{x}\PY{p}{:} \PY{n}{cos}\PY{p}{(}\PY{n}{sol\PYZus{}1}\PY{p}{)}\PY{p}{,} \PY{n}{y}\PY{p}{:} \PY{n}{sin}\PY{p}{(}\PY{n}{sol\PYZus{}1}\PY{p}{)}\PY{p}{\PYZcb{}}\PY{p}{)}\PY{p}{,} \PY{n}{f}\PY{o}{.}\PY{n}{subs}\PY{p}{(}\PY{p}{\PYZob{}}\PY{n}{x}\PY{p}{:} \PY{n}{cos}\PY{p}{(}\PY{n}{sol\PYZus{}2}\PY{p}{)}\PY{p}{,} \PY{n}{y}\PY{p}{:} \PY{n}{sin}\PY{p}{(}\PY{n}{sol\PYZus{}2}\PY{p}{)}\PY{p}{\PYZcb{}}\PY{p}{)}
\end{Verbatim}
\end{tcolorbox}

            \begin{tcolorbox}[breakable, size=fbox, boxrule=.5pt, pad at break*=1mm, opacityfill=0]
\prompt{Out}{outcolor}{167}{\boxspacing}
\begin{Verbatim}[commandchars=\\\{\}]
(-6.00968504213171, -1.51414118184529)
\end{Verbatim}
\end{tcolorbox}
        
    Our image is then \(im(f)=[-6.0096, -1.5141]\).

    \hypertarget{problem-3}{%
\subsection{Problem 3}\label{problem-3}}

    A function \(f:\mathbb{R}^2 \rightarrow \mathbb{R}\) is given by

\begin{equation*}
f(x_1 ,x_2 )=x_1^2 - 2x_1 +3x_2^5 - 5x_2^3.
\end{equation*}

\begin{enumerate}
\def\labelenumi{\alph{enumi})}
\item
  Determine all stationary points of \(f\).
\item
  State whether there is a local maximum, a local minimum, or whether
  there is a saddle point at the stationary points.
\item
  Plot for each stationary point the function along with the
  approximating polynomial \(P_{2}\) of degree (at most) \(2\).
\end{enumerate}

    As the domain \(\Gamma\) of the function
\(f: \mathbb{R}^2 \rightarrow \mathbb{R}\) is defined on the whole
\(\mathbb{R}\) we can state that the function doesn't have boundaries.
To get a feelig of how the function looks like it's been plotted below.

    \begin{tcolorbox}[breakable, size=fbox, boxrule=1pt, pad at break*=1mm,colback=cellbackground, colframe=cellborder]
\prompt{In}{incolor}{15}{\boxspacing}
\begin{Verbatim}[commandchars=\\\{\}]
\PY{n}{f} \PY{o}{=} \PY{n}{x}\PY{o}{*}\PY{o}{*}\PY{l+m+mi}{2}\PY{o}{\PYZhy{}}\PY{l+m+mi}{2}\PY{o}{*}\PY{n}{x} \PY{o}{+} \PY{l+m+mi}{3}\PY{o}{*}\PY{n}{y}\PY{o}{*}\PY{o}{*}\PY{l+m+mi}{2} \PY{o}{\PYZhy{}} \PY{l+m+mi}{5}\PY{o}{*}\PY{n}{y}\PY{o}{*}\PY{o}{*}\PY{l+m+mi}{2}
\PY{n}{dtuplot}\PY{o}{.}\PY{n}{plot3d}\PY{p}{(}\PY{n}{f}\PY{p}{,} \PY{p}{(}\PY{n}{x}\PY{p}{,} \PY{o}{\PYZhy{}}\PY{l+m+mi}{10}\PY{p}{,} \PY{l+m+mi}{10}\PY{p}{)}\PY{p}{,} \PY{p}{(}\PY{n}{y}\PY{p}{,}\PY{o}{\PYZhy{}}\PY{l+m+mi}{10}\PY{p}{,} \PY{l+m+mi}{10}\PY{p}{)}\PY{p}{)}
\end{Verbatim}
\end{tcolorbox}

    \begin{center}
    \adjustimage{max size={0.9\linewidth}{0.9\paperheight}}{output_31_0.png}
    \end{center}
    { \hspace*{\fill} \\}
    
            \begin{tcolorbox}[breakable, size=fbox, boxrule=.5pt, pad at break*=1mm, opacityfill=0]
\prompt{Out}{outcolor}{15}{\boxspacing}
\begin{Verbatim}[commandchars=\\\{\}]
<spb.backends.matplotlib.matplotlib.MatplotlibBackend at 0x76013c0f75e0>
\end{Verbatim}
\end{tcolorbox}
        
    Now we can procceed to finding the gradient and solving the system of 2
equations
\(\begin{cases} \frac{\partial f}{\partial x} = 0 \\ \frac{\partial f}{\partial y}= 0\end{cases}\)

    \begin{tcolorbox}[breakable, size=fbox, boxrule=1pt, pad at break*=1mm,colback=cellbackground, colframe=cellborder]
\prompt{In}{incolor}{16}{\boxspacing}
\begin{Verbatim}[commandchars=\\\{\}]
\PY{n}{Nf} \PY{o}{=} \PY{n}{dtutools}\PY{o}{.}\PY{n}{gradient}\PY{p}{(}\PY{n}{f}\PY{p}{)} \PY{c+c1}{\PYZsh{}finding gradient}
\PY{n}{extremums} \PY{o}{=} \PY{n}{solve}\PY{p}{(}\PY{n}{Nf}\PY{p}{,} \PY{n}{x}\PY{p}{,}\PY{n}{y}\PY{p}{)} \PY{c+c1}{\PYZsh{}solving the system}
\PY{n}{display}\PY{p}{(}\PY{n}{Nf}\PY{p}{,} \PY{n}{extremums}\PY{p}{)}
\end{Verbatim}
\end{tcolorbox}

    $\displaystyle \left[\begin{matrix}2 x - 2\\- 4 y\end{matrix}\right]$

    
    
    \begin{Verbatim}[commandchars=\\\{\}]
\{x: 1, y: 0\}
    \end{Verbatim}

    
    It is evident from gradient expression that \(f\) is differentiable on
the whole \(\Gamma\). That is the only stationary points exist for this
function are the points where the gradient equals to \(0\). And we have
already found such point: \((1,0)\). The next and the last step will be
determination if the point local maximum, minimum or it is a saddle
point.

    \begin{tcolorbox}[breakable, size=fbox, boxrule=1pt, pad at break*=1mm,colback=cellbackground, colframe=cellborder]
\prompt{In}{incolor}{17}{\boxspacing}
\begin{Verbatim}[commandchars=\\\{\}]
\PY{n}{H} \PY{o}{=} \PY{n}{dtutools}\PY{o}{.}\PY{n}{hessian}\PY{p}{(}\PY{n}{f}\PY{p}{)} \PY{c+c1}{\PYZsh{} finding a hessian matrix of f}
\PY{n}{display}\PY{p}{(}\PY{n}{H}\PY{p}{)}
\end{Verbatim}
\end{tcolorbox}

    $\displaystyle \left[\begin{matrix}2 & 0\\0 & -4\end{matrix}\right]$

    
    As the hessian matrix is a diagonal matrix and the diagonal entries have
opposite signs, it can be concluded that the function \(f\) has one
saddle point with coordinates \((1,0,-1)\).

    For convinience we will write a python function that will take a
function and expansion point and returns eproximation polynomial of the
2 degree.

    \begin{tcolorbox}[breakable, size=fbox, boxrule=1pt, pad at break*=1mm,colback=cellbackground, colframe=cellborder]
\prompt{In}{incolor}{18}{\boxspacing}
\begin{Verbatim}[commandchars=\\\{\}]
\PY{k}{def} \PY{n+nf}{P}\PY{p}{(}\PY{n}{f}\PY{p}{,} \PY{n}{x0}\PY{p}{,} \PY{n}{y0}\PY{p}{)}\PY{p}{:}
    \PY{n}{p} \PY{o}{=} \PY{p}{\PYZob{}}\PY{n}{x}\PY{p}{:} \PY{n}{x0}\PY{p}{,} \PY{n}{y}\PY{p}{:} \PY{n}{y0}\PY{p}{\PYZcb{}}
    \PY{n}{dx} \PY{o}{=} \PY{n}{Matrix}\PY{p}{(}\PY{p}{[}\PY{p}{[}\PY{n}{x}\PY{o}{\PYZhy{}}\PY{n}{x0}\PY{p}{]}\PY{p}{,}\PY{p}{[}\PY{n}{y}\PY{o}{\PYZhy{}}\PY{n}{y0}\PY{p}{]}\PY{p}{]}\PY{p}{)} \PY{c+c1}{\PYZsh{}defining x\PYZhy{}x0 vector}
    \PY{n}{Nf} \PY{o}{=} \PY{n}{dtutools}\PY{o}{.}\PY{n}{gradient}\PY{p}{(}\PY{n}{f}\PY{p}{)}\PY{o}{.}\PY{n}{subs}\PY{p}{(}\PY{n}{p}\PY{p}{)} \PY{c+c1}{\PYZsh{}finding gradient at the point (x0,y0)}
    \PY{n}{H} \PY{o}{=} \PY{n}{dtutools}\PY{o}{.}\PY{n}{hessian}\PY{p}{(}\PY{n}{f}\PY{p}{)}\PY{o}{.}\PY{n}{subs}\PY{p}{(}\PY{n}{p}\PY{p}{)} \PY{c+c1}{\PYZsh{}finding hessian matrix at the point (x0,y0)}
    \PY{k}{return} \PY{n}{f}\PY{o}{.}\PY{n}{subs}\PY{p}{(}\PY{n}{p}\PY{p}{)} \PY{o}{+} \PY{p}{(}\PY{n}{Nf}\PY{o}{.}\PY{n}{T}\PY{o}{*}\PY{n}{dx}\PY{p}{)}\PY{p}{[}\PY{l+m+mi}{0}\PY{p}{]} \PY{o}{+} \PY{l+m+mf}{0.5}\PY{o}{*}\PY{n}{dx}\PY{o}{.}\PY{n}{dot}\PY{p}{(}\PY{n}{H}\PY{o}{*}\PY{n}{dx}\PY{p}{)} \PY{c+c1}{\PYZsh{}returning taylor approxiamtion}

\PY{n}{P2} \PY{o}{=} \PY{n}{P}\PY{p}{(}\PY{n}{f}\PY{p}{,} \PY{l+m+mi}{0}\PY{p}{,} \PY{l+m+mi}{0}\PY{p}{)}
\PY{n}{display}\PY{p}{(}\PY{n}{P2}\PY{p}{,} \PY{n}{P2}\PY{o}{.}\PY{n}{subs}\PY{p}{(}\PY{p}{\PYZob{}}\PY{n}{x}\PY{p}{:} \PY{l+m+mi}{1} \PY{p}{,} \PY{n}{y}\PY{p}{:} \PY{l+m+mi}{0}\PY{p}{\PYZcb{}}\PY{p}{)}\PY{p}{)}
\end{Verbatim}
\end{tcolorbox}

    $\displaystyle 1.0 x^{2} - 2 x - 2.0 y^{2}$

    
    $\displaystyle -1.0$

    
    The approximation is obviously the same because the initial function is
a polynomial itself.

    \begin{tcolorbox}[breakable, size=fbox, boxrule=1pt, pad at break*=1mm,colback=cellbackground, colframe=cellborder]
\prompt{In}{incolor}{20}{\boxspacing}
\begin{Verbatim}[commandchars=\\\{\}]
\PY{n}{xlim} \PY{o}{=} \PY{p}{(}\PY{n}{x}\PY{p}{,} \PY{o}{\PYZhy{}}\PY{l+m+mi}{5}\PY{p}{,} \PY{l+m+mi}{5}\PY{p}{)}
\PY{n}{ylim} \PY{o}{=} \PY{p}{(}\PY{n}{y}\PY{p}{,}\PY{o}{\PYZhy{}}\PY{l+m+mi}{5}\PY{p}{,} \PY{l+m+mi}{5}\PY{p}{)}
\PY{n}{pl} \PY{o}{=} \PY{n}{dtuplot}\PY{o}{.}\PY{n}{plot3d}\PY{p}{(}\PY{n}{f}\PY{p}{,} \PY{n}{xlim}\PY{p}{,} \PY{n}{ylim}\PY{p}{,} \PY{n}{show}\PY{o}{=}\PY{k+kc}{False}\PY{p}{)} \PY{c+c1}{\PYZsh{}plot the function}
\PY{n}{pl}\PY{o}{.}\PY{n}{extend}\PY{p}{(}\PY{n}{dtuplot}\PY{o}{.}\PY{n}{plot3d}\PY{p}{(}\PY{n}{P2}\PY{p}{,} \PY{n}{xlim}\PY{p}{,} \PY{n}{ylim}\PY{p}{,} \PY{n}{show}\PY{o}{=}\PY{k+kc}{False}\PY{p}{)}\PY{p}{)} \PY{c+c1}{\PYZsh{}plot the approximating polynomial }
\PY{c+c1}{\PYZsh{}pl.extend(dtuplot.scatter(np.array([1,0,\PYZhy{}1]), rendering\PYZus{}kw=\PYZob{}\PYZdq{}color\PYZdq{}:\PYZsq{}b\PYZsq{}\PYZcb{},show=False)) \PYZsh{}the point}
\PY{n}{pl}\PY{o}{.}\PY{n}{show}\PY{p}{(}\PY{p}{)}
\end{Verbatim}
\end{tcolorbox}

    \begin{center}
    \adjustimage{max size={0.9\linewidth}{0.9\paperheight}}{output_40_0.png}
    \end{center}
    { \hspace*{\fill} \\}
    
    \hypertarget{problem-4}{%
\subsection{Problem 4}\label{problem-4}}

    A function \(f:[0,5] \rightarrow \mathbb{R}\) is given by the expression
\(f(x)=2x+3\).

\begin{enumerate}
\def\labelenumi{\alph{enumi})}
\item
  Determine, using Python, a value of the Riemann sum over the interval
  \([0,5]\) with 30 subintervals, where firstly the left interval
  end-points of the subintervals are used. Repeat with the right
  interval end-points.
\item
  Determine exact values of the Riemann sum over the interval \([0,5]\)
  with \(n\) subintervals, where firstly the left interval end-points of
  the subintervals are used. Repeat with the right interval end-points.
\item
  Provide an exact expression for the maximal and minimal errors. Both
  expressions may only depend on \(n\).
\item
  Argue in this special case for the fact that the Riemann sum has the
  same limit value regardless of the choice of point in the
  subintervals.
\end{enumerate}

    \begin{tcolorbox}[breakable, size=fbox, boxrule=1pt, pad at break*=1mm,colback=cellbackground, colframe=cellborder]
\prompt{In}{incolor}{5}{\boxspacing}
\begin{Verbatim}[commandchars=\\\{\}]
\PY{n}{f} \PY{o}{=} \PY{l+m+mi}{2}\PY{o}{*}\PY{n}{x} \PY{o}{+} \PY{l+m+mi}{3} \PY{c+c1}{\PYZsh{}defiing the function}
\end{Verbatim}
\end{tcolorbox}

    The Riemann Sum on the interval \(x \in [a,b]\) can be defined as
\(\sum_{i=1}^{n} f(x_i)\triangle x\), where
\(\triangle x = \frac{b-a}{n}\). \(x_i\) that is at the beginning of
\(\triangle x\) interval defines the left sum and \(x_i\) that is at the
end of \(\triangle x\) interval defines the right sum. For the left sum
\(x_i = a + i\triangle x, \, i = 0, 1, ... ,n-1\), for the right sum
\(x_i = a + i\triangle x, \, i = 1, ... ,n\)

    \begin{tcolorbox}[breakable, size=fbox, boxrule=1pt, pad at break*=1mm,colback=cellbackground, colframe=cellborder]
\prompt{In}{incolor}{9}{\boxspacing}
\begin{Verbatim}[commandchars=\\\{\}]
\PY{k}{def} \PY{n+nf}{riemann\PYZus{}right}\PY{p}{(}\PY{n}{f}\PY{p}{,} \PY{n}{interval}\PY{p}{,} \PY{n}{n}\PY{p}{)}\PY{p}{:}
    \PY{n}{dx} \PY{o}{=} \PY{p}{(}\PY{n}{interval}\PY{p}{[}\PY{l+m+mi}{1}\PY{p}{]}\PY{o}{\PYZhy{}}\PY{n}{interval}\PY{p}{[}\PY{l+m+mi}{0}\PY{p}{]}\PY{p}{)}\PY{o}{/}\PY{n}{n} \PY{c+c1}{\PYZsh{}defining dx}
    \PY{n}{s} \PY{o}{=} \PY{l+m+mi}{0} \PY{c+c1}{\PYZsh{} initial sum equals 0}
    \PY{k}{for} \PY{n}{i} \PY{o+ow}{in} \PY{n+nb}{range}\PY{p}{(}\PY{n}{n}\PY{p}{)}\PY{p}{:}
        \PY{n}{s} \PY{o}{+}\PY{o}{=} \PY{n}{f}\PY{o}{.}\PY{n}{subs}\PY{p}{(}\PY{n}{x}\PY{p}{,} \PY{n}{interval}\PY{p}{[}\PY{l+m+mi}{0}\PY{p}{]} \PY{o}{+} \PY{p}{(}\PY{n}{i}\PY{o}{+}\PY{l+m+mi}{1}\PY{p}{)}\PY{o}{*}\PY{n}{dx}\PY{p}{)}\PY{o}{*}\PY{n}{dx}  \PY{c+c1}{\PYZsh{}calculating and adding f(xi)dx to the sum}
    \PY{k}{return} \PY{n}{s}
\PY{k}{def} \PY{n+nf}{riemann\PYZus{}left}\PY{p}{(}\PY{n}{f}\PY{p}{,} \PY{n}{interval}\PY{p}{,} \PY{n}{n}\PY{p}{)}\PY{p}{:}
    \PY{n}{dx} \PY{o}{=} \PY{p}{(}\PY{n}{interval}\PY{p}{[}\PY{l+m+mi}{1}\PY{p}{]}\PY{o}{\PYZhy{}}\PY{n}{interval}\PY{p}{[}\PY{l+m+mi}{0}\PY{p}{]}\PY{p}{)}\PY{o}{/}\PY{n}{n}  \PY{c+c1}{\PYZsh{}defining dx}
    \PY{n}{s} \PY{o}{=} \PY{l+m+mi}{0}   \PY{c+c1}{\PYZsh{} initial sum equals 0}
    \PY{k}{for} \PY{n}{i} \PY{o+ow}{in} \PY{n+nb}{range}\PY{p}{(}\PY{n}{n}\PY{p}{)}\PY{p}{:}
        \PY{n}{s} \PY{o}{+}\PY{o}{=} \PY{n}{f}\PY{o}{.}\PY{n}{subs}\PY{p}{(}\PY{n}{x}\PY{p}{,} \PY{n}{interval}\PY{p}{[}\PY{l+m+mi}{0}\PY{p}{]} \PY{o}{+} \PY{n}{i}\PY{o}{*}\PY{n}{dx}\PY{p}{)}\PY{o}{*}\PY{n}{dx} \PY{c+c1}{\PYZsh{}calculating and adding f(xi)dx to the sum}
    \PY{k}{return} \PY{n}{s}

\PY{n}{rsum} \PY{o}{=} \PY{n}{riemann\PYZus{}right}\PY{p}{(}\PY{n}{f}\PY{p}{,} \PY{p}{[}\PY{l+m+mi}{0}\PY{p}{,}\PY{l+m+mi}{5}\PY{p}{]}\PY{p}{,}\PY{l+m+mi}{30}\PY{p}{)}
\PY{n}{lsum} \PY{o}{=} \PY{n}{riemann\PYZus{}left}\PY{p}{(}\PY{n}{f}\PY{p}{,} \PY{p}{[}\PY{l+m+mi}{0}\PY{p}{,}\PY{l+m+mi}{5}\PY{p}{]}\PY{p}{,} \PY{l+m+mi}{30}\PY{p}{)}

\PY{n}{display}\PY{p}{(}\PY{n}{rsum}\PY{p}{,} \PY{n}{lsum}\PY{p}{)}
\end{Verbatim}
\end{tcolorbox}

    $\displaystyle 40.8333333333333$

    
    $\displaystyle 39.1666666666667$

    
    \begin{tcolorbox}[breakable, size=fbox, boxrule=1pt, pad at break*=1mm,colback=cellbackground, colframe=cellborder]
\prompt{In}{incolor}{5}{\boxspacing}
\begin{Verbatim}[commandchars=\\\{\}]
\PY{n}{display}\PY{p}{(}\PY{n}{riemann\PYZus{}left}\PY{p}{(}\PY{n}{f}\PY{p}{,} \PY{p}{[}\PY{l+m+mi}{0}\PY{p}{,}\PY{l+m+mi}{5}\PY{p}{]}\PY{p}{,} \PY{l+m+mi}{300000}\PY{p}{)}\PY{p}{,} \PY{n}{riemann\PYZus{}right}\PY{p}{(}\PY{n}{f}\PY{p}{,} \PY{p}{[}\PY{l+m+mi}{0}\PY{p}{,}\PY{l+m+mi}{5}\PY{p}{]}\PY{p}{,}\PY{l+m+mi}{300000}\PY{p}{)}\PY{p}{)}
\end{Verbatim}
\end{tcolorbox}

    $\displaystyle 39.9999166666667$

    
    $\displaystyle 40.0000833333333$

    
    The Riemann Sum upproaches the true value of the integral
\(\int_{0}^{5} 2x - 3 \, dx\) as \(n \rightarrow \infty\) or
alternatively area under the line.

    The maximum error of the following method can be estimated by the
\(\triangle x|f(b) - f(a)| = \frac{b-a}{n}|f(b)-f(a)| = \frac{50}{n}\).
That is essentialy the difference between the right and the left sum. We
can check it using the results above.

    \begin{tcolorbox}[breakable, size=fbox, boxrule=1pt, pad at break*=1mm,colback=cellbackground, colframe=cellborder]
\prompt{In}{incolor}{11}{\boxspacing}
\begin{Verbatim}[commandchars=\\\{\}]
\PY{n}{max\PYZus{}err} \PY{o}{=} \PY{l+m+mi}{50}\PY{o}{/}\PY{l+m+mi}{30}
\PY{n}{display}\PY{p}{(}\PY{n}{rsum} \PY{o}{\PYZhy{}} \PY{n}{lsum}\PY{p}{,} \PY{n}{max\PYZus{}err}\PY{p}{)}
\end{Verbatim}
\end{tcolorbox}

    $\displaystyle 1.66666666666668$

    
    
    \begin{Verbatim}[commandchars=\\\{\}]
1.6666666666666667
    \end{Verbatim}

    
    The minimum error will be the actual difference between the true area
and the approximated one.

    \begin{tcolorbox}[breakable, size=fbox, boxrule=1pt, pad at break*=1mm,colback=cellbackground, colframe=cellborder]
\prompt{In}{incolor}{12}{\boxspacing}
\begin{Verbatim}[commandchars=\\\{\}]
\PY{n}{min\PYZus{}err} \PY{o}{=} \PY{n}{rsum} \PY{o}{\PYZhy{}} \PY{l+m+mi}{40}
\PY{n}{min\PYZus{}err}
\end{Verbatim}
\end{tcolorbox}
 
            
\prompt{Out}{outcolor}{12}{}
    
    $\displaystyle 0.833333333333336$

    

    As the function is linear, the slope is constant on the whole interval
\([a,b]\), and the error of the upper sum will be the same as that of
the left sum.

\(\textit{Proof}\)

From the geometric consideration, the error can be estimated as the sum
of triangles
\(\epsilon = \frac{1}{2}n \triangle x^2 f^{'}(x) = n(\frac{b-a}{n})^2 = \frac{(b-a)^2}{n} = \frac{25}{n}\),
what is exactly half of the maximum error.

    \begin{tcolorbox}[breakable, size=fbox, boxrule=1pt, pad at break*=1mm,colback=cellbackground, colframe=cellborder]
\prompt{In}{incolor}{15}{\boxspacing}
\begin{Verbatim}[commandchars=\\\{\}]
\PY{n}{calc\PYZus{}err} \PY{o}{=} \PY{l+m+mi}{25}\PY{o}{/}\PY{l+m+mi}{30}
\PY{n}{display}\PY{p}{(}\PY{n}{min\PYZus{}err}\PY{p}{,} \PY{l+m+mi}{40}\PY{o}{\PYZhy{}}\PY{n}{lsum}\PY{p}{,} \PY{n}{calc\PYZus{}err}\PY{p}{)} \PY{c+c1}{\PYZsh{}assesing the difference between the left/right sum and the true area}
\end{Verbatim}
\end{tcolorbox}

    $\displaystyle 0.833333333333336$

    
    $\displaystyle 0.833333333333343$

    
    
    \begin{Verbatim}[commandchars=\\\{\}]
0.8333333333333334
    \end{Verbatim}

    
    The differences are equal to each other and equal to the calculated
above \(\epsilon = \frac{25}{30}\).

    The image below visualizes left(blue) and right(orange) sums.

    \begin{tcolorbox}[breakable, size=fbox, boxrule=1pt, pad at break*=1mm,colback=cellbackground, colframe=cellborder]
\prompt{In}{incolor}{16}{\boxspacing}
\begin{Verbatim}[commandchars=\\\{\}]
\PY{n}{Image}\PY{p}{(}\PY{n}{filename}\PY{o}{=}\PY{l+s+s2}{\PYZdq{}}\PY{l+s+s2}{zoom.png}\PY{l+s+s2}{\PYZdq{}}\PY{p}{,} \PY{n}{width}\PY{o}{=}\PY{l+m+mi}{200}\PY{p}{,} \PY{n}{height}\PY{o}{=}\PY{l+m+mi}{100}\PY{p}{)}
\end{Verbatim}
\end{tcolorbox}
 
            
\prompt{Out}{outcolor}{16}{}
    
    \begin{center}
    \adjustimage{max size={0.9\linewidth}{0.9\paperheight}}{output_56_0.png}
    \end{center}
    { \hspace*{\fill} \\}
    


    % Add a bibliography block to the postdoc
    
    
    
\end{document}
